\documentclass[12pt]{article}

\usepackage{float}

\usepackage{standalone}

%\usepackage[utf8x]{inputenc}

%%% PAGE DIMENSIONS
\usepackage{geometry}
\geometry{a4paper}
\geometry{margin=2.54cm} % for example, change the margins to 2 inches all round

\usepackage{graphicx} % support the \includegraphics command and options

\usepackage[parfill]{parskip} % Activate to begin paragraphs with an empty line rather than an indent

%%% PACKAGES
\usepackage{booktabs} % for much better looking tables
\usepackage{array} % for better arrays (e.g., matrices) in maths
\usepackage{paralist} % very flexible & customisable lists (e.g., enumerate/itemise, etc.)
\usepackage{verbatim} % adds environment for commenting out blocks of text & for better verbatim
\usepackage{subfig} % make it possible to include more than one captioned figure/table in a single float
% These packages are all incorporated in the memoir class to one degree or another...

\usepackage{multicol}
\usepackage{multirow}
\usepackage{xcolor}
\usepackage{amsmath}

\usepackage[T1]{fontenc}
\usepackage{lmodern}

% Sans-serif font
\renewcommand{\familydefault}{\sfdefault}

\usepackage{makecell}

\renewcommand{\arraystretch}{1.1}

%%% HEADERS & FOOTERS
\usepackage{fancyhdr} % This should be set AFTER setting up the page geometry
\pagestyle{fancy} % options: empty , plain , fancy
\fancyhead[L]{\leftmark}
\fancyhead[C]{}
\fancyhead[R]{\rightmark}
\fancyfoot[L]{}
\fancyfoot[C]{}
\fancyfoot[R]{\thepage}
\renewcommand{\headrulewidth}{0pt}
\renewcommand{\footrulewidth}{0pt}
\setlength{\headheight}{52pt}

\fancypagestyle{plain}{
	\fancyhf{} % clear all header and footer fields
	\fancyfoot[R]{\thepage} % except the center
	\renewcommand{\headrulewidth}{0pt}
	\renewcommand{\footrulewidth}{0pt}
}

%%% BIBLIOGRAPHY
\usepackage[numbers]{natbib}
\bibliographystyle{vancouver}

%%% SECTION TITLE APPEARANCE
\usepackage{sectsty}
\allsectionsfont{\sffamily\mdseries\upshape}

%%% ToC (table of contents) APPEARANCE
%\usepackage[nottoc,notlof,notlot]{tocbibind} % Put the bibliography in the ToC
%\usepackage[titles,subfigure]{tocloft} % Alter the style of the Table of Contents
%\renewcommand{\cftsecfont}{\rmfamily\mdseries\upshape}
%\renewcommand{\cftsecpagefont}{\rmfamily\mdseries\upshape} % No bold!

\usepackage[bookmarks,bookmarksnumbered,bookmarksopen,hidelinks]{hyperref}

\usepackage{bookmark}

% TITLE
\title{NIH HCW Study}
\author{Arseniy Khvorov}
\begin{document}

\maketitle

\section{Data}

The study runs over 4 years (2020 to 2023). 
Participants are recruited every year.
Participants are followed up until they drop out or the study ends.

Flu vaccination status in the 5 years prior to recruitment is collected. Vaccination status cannot always be obtained for every relevant year. Table \ref{tab:missing-vachist} contains counts of participants split by the number of years we have their vaccination history for.

\begin{table}

\caption{\label{tab:missing-vachist}Counts of participants split by the number of years for which we have their vaccination history. 
            "Ref year" is the reference year the 5 prior years' vaccination history relates to.
            For example, if the reference year is 2022 then the 5 years prior to that are 2021, 2020, 2019, 2018 and 2017.
            If we know someone's vaccination status in 2021, 2020, 2019 but not in 2018 and 2017 then they would have 3 "prior years with known status".
            This participant would then be counted in the "Total", "At least 1", "At least 2" and "At least 3" rows but not in "At least 4" and "All 5" rows.}
\centering
\begin{tabular}[t]{l>{\raggedright\arraybackslash}p{2cm}>{\raggedleft\arraybackslash}p{1cm}>{\raggedleft\arraybackslash}p{1cm}>{\raggedleft\arraybackslash}p{1cm}r}
\toprule
\multicolumn{2}{c}{ } & \multicolumn{3}{c}{Recruitment year} \\
\cmidrule(l{3pt}r{3pt}){3-5}
Ref year & Prior years with known status & 2020 & 2021 & 2022 & Total\\
\midrule
 & All 5 & 507 & 0 & 0 & 507\\

 & At least 4 & 548 & 610 & 43 & 1201\\

 & At least 3 & 593 & 677 & 418 & 1688\\

 & At least 2 & 627 & 723 & 437 & 1787\\

 & At least 1 & 636 & 750 & 458 & 1844\\

\multirow{-6}{*}{\raggedright\arraybackslash 2020} & Total & 637 & 759 & 479 & 1875\\
\cmidrule{1-6}
 & All 5 & 530 & 606 & 43 & 1179\\

 & At least 4 & 584 & 676 & 414 & 1674\\

 & At least 3 & 626 & 723 & 436 & 1785\\

 & At least 2 & 636 & 749 & 456 & 1841\\

 & At least 1 & 637 & 756 & 469 & 1862\\

\multirow{-6}{*}{\raggedright\arraybackslash 2021} & Total & 637 & 759 & 479 & 1875\\
\cmidrule{1-6}
 & All 5 & 342 & 653 & 412 & 1407\\

 & At least 4 & 595 & 717 & 435 & 1747\\

 & At least 3 & 630 & 745 & 456 & 1831\\

 & At least 2 & 637 & 755 & 468 & 1860\\

 & At least 1 & 637 & 759 & 475 & 1871\\

\multirow{-6}{*}{\raggedright\arraybackslash 2022} & Total & 637 & 759 & 479 & 1875\\
\cmidrule{1-6}
 & All 5 & 274 & 407 & 424 & 1105\\

 & At least 4 & 377 & 721 & 454 & 1552\\

 & At least 3 & 627 & 748 & 468 & 1843\\

 & At least 2 & 635 & 755 & 475 & 1865\\

 & At least 1 & 637 & 759 & 479 & 1875\\

\multirow{-6}{*}{\raggedright\arraybackslash 2023} & Total & 637 & 759 & 479 & 1875\\
\bottomrule
\end{tabular}
\end{table}


Participants are bled every year at before flu vaccination (day 0), 14 days after flu vaccination and at the end of the flu season (approximately 220 days after flu vaccination). 
Some participants have an extra bleed 7 days post-vaccination.
See Table \ref{tab:routine-bleed-counts} for counts of these routine bleeds over the study years. 
See Figure \ref{fig:bleed-dates} for bleed dates.
Note that not all "day 7/14/220 bleeds" are actually 7/14/220 days apart from vaccination and not all baseline (day 0) bleeds happened on the same date as vaccination.
I call them day 0/7/14/220 bleeds for convenience.
See Figure \ref{fig:bleed-intervals} for the distribution of actual bleed intervals for every timepoint.

\begin{table}

\caption{\label{tab:routine-bleed-counts}Counts of bleeds with measured antibody titres for each timepoint (day post-vaccination) 
        over the study years for different prior vaccination groups.}
\centering
\begin{tabular}[t]{ll>{\raggedleft\arraybackslash}p{1cm}>{\raggedleft\arraybackslash}p{1cm}>{\raggedleft\arraybackslash}p{1cm}>{\raggedleft\arraybackslash}p{1cm}>{\raggedleft\arraybackslash}p{1cm}>{\raggedleft\arraybackslash}p{1cm}r}
\toprule
\multicolumn{2}{c}{ } & \multicolumn{6}{c}{Known vaccinations in the 5 years before bleed} \\
\cmidrule(l{3pt}r{3pt}){3-8}
Day & Year & 0 & 1 & 2 & 3 & 4 & 5 & Total\\
\midrule
 & 2020 & 74 & 70 & 71 & 65 & 74 & 265 & 619\\

 & 2021 & 40 & 69 & 90 & 117 & 141 & 662 & 1119\\

\multirow{-3}{*}{\raggedright\arraybackslash 0} & Total & 114 & 139 & 161 & 182 & 215 & 927 & 1738\\
\cmidrule{1-9}
 & 2020 & 40 & 0 & 0 & 0 & 0 & 62 & 102\\

 & 2021 & 11 & 16 & 0 & 0 & 0 & 61 & 88\\

\multirow{-3}{*}{\raggedright\arraybackslash 7} & Total & 51 & 16 & 0 & 0 & 0 & 123 & 190\\
\cmidrule{1-9}
 & 2020 & 54 & 70 & 69 & 67 & 75 & 262 & 597\\

 & 2021 & 15 & 56 & 82 & 115 & 135 & 628 & 1031\\

\multirow{-3}{*}{\raggedright\arraybackslash 14} & Total & 69 & 126 & 151 & 182 & 210 & 890 & 1628\\
\cmidrule{1-9}
 & 2020 & 69 & 66 & 66 & 62 & 72 & 253 & 588\\

 & 2021 & 34 & 58 & 84 & 112 & 137 & 619 & 1044\\

\multirow{-3}{*}{\raggedright\arraybackslash 220} & Total & 103 & 124 & 150 & 174 & 209 & 872 & 1632\\
\cmidrule{1-9}
 & 2020 & 237 & 206 & 206 & 194 & 221 & 842 & 1906\\

 & 2021 & 100 & 199 & 256 & 344 & 413 & 1970 & 3282\\

\multirow{-3}{*}{\raggedright\arraybackslash Total} & Total & 337 & 405 & 462 & 538 & 634 & 2812 & 5188\\
\bottomrule
\end{tabular}
\end{table}


\begin{figure}
	\includegraphics[width=\textwidth,height=\textheight,keepaspectratio]{figure-bleed-dates/figure-bleed-dates.pdf}
	\caption{Bleed dates over the study years. Each row is an individual.}
	\label{fig:bleed-dates}
\end{figure}

\begin{figure}
	\includegraphics[width=\textwidth,height=\textheight,keepaspectratio]{figure-bleed-dates/figure-bleed-intervals.pdf}
	\caption{Actual intervals for every timepoint.}
	\label{fig:bleed-intervals}
\end{figure}

Participants are followed up for respiratory symptoms over the flu season and are bled after PCR-confirmed flu infections.
See Table \ref{tab:infection-counts} for counts of all recorded infections and post-infection bleeds.
See Figure \ref{fig:postinf-bleed-dates} for all recorded infections and associated bleed dates.

\begin{table}

\caption{\label{tab:infection-counts}Counts of positive swabs (with their test result) and post-infection bleeds.
        If a swab had two (or more) positive results it does not count twice but instead its
        results are merged and comma-separated (e.g., "Parainfluenza, Piconavirus")}
\centering
\begin{tabular}[t]{lrrrr}
\toprule
  & 2020 & 2021 & 2022 & Total\\
\midrule
Post-infection bleed day 7 & 0 & 0 & 47 & 47\\
Post-infection bleed day 14 & 0 & 0 & 106 & 106\\
Post-infection bleed day 30 & 0 & 0 & 84 & 84\\
\addlinespace
Total bleeds & 0 & 0 & 237 & 237\\
\addlinespace
Flu A (unsubtyped) & 0 & 0 & 32 & 32\\
Flu A H3 & 0 & 0 & 5 & 5\\
\addlinespace
SARS-CoV-2 & 0 & 7 & 243 & 250\\
\addlinespace
Adenovirus & 0 & 1 & 0 & 1\\
Metapneumovirus & 0 & 6 & 8 & 14\\
Piconavirus & 0 & 47 & 40 & 87\\
Parainfluenza & 0 & 13 & 4 & 17\\
Other & 9 & 67 & 70 & 146\\
\addlinespace
Flu A (unsubtyped), Flu A H3 & 0 & 0 & 2 & 2\\
Flu A (unsubtyped), Piconavirus & 0 & 0 & 2 & 2\\
Flu A (unsubtyped), Other & 0 & 0 & 2 & 2\\
\addlinespace
SARS-CoV-2, Piconavirus & 0 & 0 & 1 & 1\\
SARS-CoV-2, Other & 0 & 0 & 1 & 1\\
\addlinespace
Metapneumovirus, Piconavirus & 0 & 0 & 1 & 1\\
Parainfluenza, Piconavirus & 0 & 0 & 2 & 2\\
Piconavirus, Other & 0 & 1 & 1 & 2\\
\addlinespace
Total flu & 0 & 0 & 43 & 43\\
Total SARS-CoV-2 & 0 & 7 & 245 & 252\\
Total non-flu non-covid & 9 & 135 & 126 & 270\\
\addlinespace
Total positive & 9 & 142 & 414 & 565\\
\bottomrule
\end{tabular}
\end{table}


\begin{figure}
	\includegraphics[width=\textwidth,height=\textheight,keepaspectratio]{infection-dates-counts/figure-infection-dates.pdf}
	\caption{Dates of nasal swab collections that were positive for one of the viruses in the test panel. Also included are dates of post-infection bleeds. Each row is an individual.}
	\label{fig:postinf-bleed-dates}
\end{figure}

Average titres for day 0, 14 and 220 against vaccine strain viruses split by vaccination status are in Figure \ref{fig:titre-averages-H1} (A H1), \ref{fig:titre-averages-H3} (A H3), \ref{fig:titre-averages-BVic} (B Victoria) and \ref{fig:titre-averages-BYam} (B Yamagata).

\begin{figure}
	\includegraphics[width=\textwidth,height=\textheight,keepaspectratio]{figure-titre/averages-H1.pdf}
	\caption{A H1 titre averages.}
	\label{fig:titre-averages-H1}
\end{figure}

\begin{figure}
	\includegraphics[width=\textwidth,height=\textheight,keepaspectratio]{figure-titre/averages-H3.pdf}
	\caption{A H3 titre averages.}
	\label{fig:titre-averages-H3}
\end{figure}

\begin{figure}
	\includegraphics[width=\textwidth,height=\textheight,keepaspectratio]{figure-titre/averages-BVic.pdf}
	\caption{B Victoria titre averages.}
	\label{fig:titre-averages-BVic}
\end{figure}

\begin{figure}
	\includegraphics[width=\textwidth,height=\textheight,keepaspectratio]{figure-titre/averages-BYam.pdf}
	\caption{B Yamagata titre averages.}
	\label{fig:titre-averages-BYam}
\end{figure}

\end{document}
